\documentclass[aspectratio=169]{beamer}
\usetheme{m}  % Use metropolis theme
\title{Diskrete Strukturen \newline Praktika 2}
\date{\today}
\author{Liu Kin \newline Wombacher Sascha \newline}
\usepackage{float}
\usepackage{listings}
\usepackage[utf8]{inputenc}
\usepackage[ngerman]{babel}
\usepackage{dirtytalk} % simple quotes


\lstset{
	language=C++,
    tabsize=4,
    keepspaces,
    extendedchars=true,
    rulecolor=\color{black},
    basicstyle=\footnotesize,
    aboveskip=5pt,
    upquote=true,
    columns=fixed,
    showstringspaces=false,
    extendedchars=true,
    breaklines=true,
    frame=single,
%    showtabs=true,
    showspaces=false,
    showstringspaces=false,
    basicstyle=\tiny,
    keywordstyle=\color{blue}
}

\begin{document}

  \lstdefinestyle{numbers}
  {numbers=left, stepnumber=1, numberstyle=\tiny, numbersep=10pt}
  \lstdefinestyle{nonumbers}
  {numbers=none}

  \maketitle
  \begin{frame}{Agenda}
    \setbeamertemplate{section in toc}[sections numbered]
    \tableofcontents[hideallsubsections]
  \end{frame}
  
  
\section{Aufgabe 1}
    \begin{frame}{Aufgabe 1 \newline Multiplikation}
    	Eingabe: Ursprüngliche Permutation $x$ und eine darauf anzuwendende Permutation $P$ \newline
    	Ausgabe: Permutiertes Objekt $y$ \newline
    	\mbox{} \newline
    	Iteriere über alle Elemente in $P$
    	\begin{itemize}
    		\item setze $y[i] = x[P[i]]$
    	\end{itemize}
    \end{frame}
    
    \begin{frame}{Aufgabe 1 \newline Inversion}
    	Gleich wie die Multiplikation, \newline vertausche lediglich den Index mit dem Wert
    \end{frame}
    
    \begin{frame}{Aufgabe 1 \newline Zyklennotation}
    	Ausgabe in Zyklen-Notation
    	\begin{itemize}
    		\item Finde den nächsten (unbenutzten) Zykel/Element
    		\item Sortiere den gefundenen Zykel (kleinste Element nach vorne)
    		\item Ist die Elementanzahl im $Zykel > 1$ gib den Zykel aus
    	\end{itemize}
    \end{frame}
    
\section{Aufgabe 2}
	\begin{frame}{Gruppengenerierung}
		Zur Gruppengeneration wird jeder mit jeden Permutation multipliziert bis keine neuen Permutationen hinzugefügt werden
	\end{frame}

\section{Aufgabe 3}
	\begin{frame}{Reguläres Dreieck}
		Für ein reguläres Dreieck wurden folgende Permutationsbasen verwendet:
		\begin{itemize}
			\item Achsenspiegelung \newline \{0,2,1\}
			\item Rotation \newline \{1,2,0\}
		\end{itemize}
	\end{frame}
	
	\begin{frame}{Generelles Rechteck}
		Für ein generelles Rechteck wurden folgende Permutationsbasen verwendet:
		\begin{itemize}
			\item Achsenspiegelung an der x-Achse \newline \{2,3,0,1\}
			\item Achsenspiegelung an der y-Achse \newline \{1,0,3,2\}
			\item Rotation \newline \{3,2,1,0\}
		\end{itemize}
	\end{frame}
	
	\begin{frame}{Tetraeder}
		Für das Tetraeder wurden folgende Permutationsbasen verwendet:
		\begin{itemize}
			\item Rotation einer Fläche \newline \{0,2,3,1\}
			\item Achsenspiegelung an der 2,3-Fläche \newline \{1,0,2,3\}
		\end{itemize}
	\end{frame}
	
	\begin{frame}{Würfel}
		Für einen Würfel wurden folgende Permutationsbasen verwendet:
		\begin{itemize}
			\item Rotation um die x-Achse \newline \{1,2,3,0,5,6,7,4\}
			\item Rotation um die y-Achse \newline \{4,5,1,0,7,6,2,3\}
			\item Spiegelung an der verschobenen xz-Ebene \newline \{1,0,3,2,5,4,7,6\}
		\end{itemize}
	\end{frame}
	
	\begin{frame}{Ikosaeder}
		Für das Ikosaeder wurden folgende Permutationsbasen verwendet:
		\begin{itemize}
			\item Rotation um 0,11-Achse \newline \{0,2,3,4,5,1,7,8,9,10,6,11\}
			\item Rotation um 3,6-Achse \newline \{2,7,8,3,0,1,6,11,9,4,5,10\}
			\item Spiegelung an der 0,2,10,11-Fläche \newline \{0,3,2,1,5,4,9,8,7,6,10,11\}
		\end{itemize}
	\end{frame}

\section*{Haben Sie Fragen?}
\section*{Vielen Dank für die Aufmerksamkeit!}
  
\end{document}
