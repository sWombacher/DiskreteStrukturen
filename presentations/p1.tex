\documentclass{beamer}
\usetheme{m}  % Use metropolis theme
\title{Diskrete Strukturen \newline Praktika 1}
\date{\today}
\author{Liu Kin \newline Wombacher Sascha}
\usepackage{float}
\usepackage{listings}
\usepackage[utf8]{inputenc}
\usepackage[ngerman]{babel}



\lstset{
	language=C++,
    tabsize=4,
    keepspaces,
    extendedchars=true,
    rulecolor=\color{black},
    basicstyle=\footnotesize,
    aboveskip=5pt,
    upquote=true,
    columns=fixed,
    showstringspaces=false,
    extendedchars=true,
    breaklines=true,
    frame=single,
%    showtabs=true,
    showspaces=false,
    showstringspaces=false,
    basicstyle=\tiny,
    keywordstyle=\color{blue}
}

\begin{document}

  \lstdefinestyle{numbers}
  {numbers=left, stepnumber=1, numberstyle=\tiny, numbersep=10pt}
  \lstdefinestyle{nonumbers}
  {numbers=none}


  \maketitle
  \begin{frame}{Agenda}
    \setbeamertemplate{section in toc}[sections numbered]
    \tableofcontents[hideallsubsections]
  \end{frame}
  
  
  \section{Generatormatrix Allgemein}
    \begin{frame}{Generatormatrix}
	\begin{itemize}
		\item Sind $k \times n$ Matrizen
		\item Sind Matrizen mit linear unabhängigen Zeilen
		\item Werden verwendet um eine Nachricht der länge $k$ in ein Codewort der länge $n$ zu überführen (Matrixmultiplikation)
		\item Jede Zeile ist auch ein Codewort
  	\end{itemize}
    \end{frame}  

\section*{Vielen Dank für die Aufmerksamkeit!}
  
\end{document}
