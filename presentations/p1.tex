\documentclass[aspectratio=169]{beamer}
\usetheme{m}  % Use metropolis theme
\title{Diskrete Strukturen \newline Praktika 1}
\date{\today}
\author{Liu Kin \newline Wombacher Sascha \newline}
\usepackage{float}
\usepackage{listings}
\usepackage[utf8]{inputenc}
\usepackage[ngerman]{babel}
\usepackage{dirtytalk} % simple quotes


\lstset{
	language=C++,
    tabsize=4,
    keepspaces,
    extendedchars=true,
    rulecolor=\color{black},
    basicstyle=\footnotesize,
    aboveskip=5pt,
    upquote=true,
    columns=fixed,
    showstringspaces=false,
    extendedchars=true,
    breaklines=true,
    frame=single,
%    showtabs=true,
    showspaces=false,
    showstringspaces=false,
    basicstyle=\tiny,
    keywordstyle=\color{blue}
}

\begin{document}

  \lstdefinestyle{numbers}
  {numbers=left, stepnumber=1, numberstyle=\tiny, numbersep=10pt}
  \lstdefinestyle{nonumbers}
  {numbers=none}


  \maketitle
  \begin{frame}{Agenda}
    \setbeamertemplate{section in toc}[sections numbered]
    \tableofcontents[hideallsubsections]
  \end{frame}
  
  
\section{Allgemeines}
    \begin{frame}{Allgemeines}
    Allgemeines:
	\begin{itemize}
		\item Aufgabe 1 ist lediglich eine Festlegung von $q=2$
		\item rank/unrank/successor werden für allgemeine $q$ erklärt
		\item Wenn $q$ einen geraden Wert besitzt ist es auch ein zyklischer Code
  	\end{itemize}
	Anmerkung: \newline
	Rekursionsgleichung wurde aus dem Skript übernommen (möglicherweise ein Schreibfehler in der Aufgabenstellung!)
    \end{frame}
    
\section{Rank}
    \begin{frame}{Rank \newline Part 1: Allgemeines}
    Allgemeines:
    \begin{itemize}
	    \item Dient zur Bestimmung des \say{Index} eines übergebenen Codes
	    \item Benötigt wissen über das Alphabet $q$
	    \item Benötigt wissen über die Länge des Codes
    \end{itemize}
    \end{frame}
    
    \begin{frame}{Rank \newline Part 2: Algorithmenbeschreibung}
    Algorithmus:
	\begin{itemize}
	    \item Beginne von rechts
	    \item Gelesenes Symbol beschreibt einen Bereich
	    \item Dieser Bereich beschreibt ein Offset für den Rückgabe Index
	    \item Rekursiver Aufruf ohne des Gelesen Symbols \newline
	    \item Rückgabe bei Geradem Symbol (\say{Leserichtung von links}): \newline Offset + RekursiverRückgabewert \newline
		\item Rückgabe bei Ungeradem Symbol (\say{Leserichtung von rechts}): \newline Offset - RekursiverRückgabewert - 1 + <BereichGröße>
    \end{itemize}
    \end{frame}

	\begin{frame}{Rank \newline Part 3: Beispiel}
	Übergebenes Objekt: $rank(012)$, $q=3$ \newline
	
	Lese letztes Element: 2 (gerade)\newline
	Rückgabe: $2 \cdot q^2 + rank(01) = 23$\newline
	
	Lese letztes Element: 1 (ungerade) \newline
	Rückgabe: $1 \cdot q^1 - rank(0) - 1 + q^1 = 5$\newline
	
	Lese letztes Element: 0 (gerade, kein folge Element) \newline
	Rückgabe: $0$
	\end{frame}


\section{Unrank}
    \begin{frame}{Unrank \newline Part 1: Allgemeines}
    Allgemeines:
    \begin{itemize}
	    \item Dient zur Bestimmung eines Codes anhand eines \say{Index}
	    \item Benötigt wissen über das Alphabet $q$
	    \item Benötigt wissen über die Länge des Codes
    \end{itemize}
    \end{frame}
    
    \begin{frame}{Unrank \newline Part 2: Algorithmenbeschreibung}
    Algorithmus:
	\begin{itemize}
	    \item Beginne von rechts das Array zu füllen
	    \item Generiere $q$ Bereiche über $q^{bitPosition}$ Werte
	    \item Schreibe die Bereich Position an die nächste Array Position
	    \item Passe den Index (für rekursion) an
	    \item Ist der Bereich ungerade \say{invertiere} alle nachfolge Werte \newline invertiere: Wert = q - 1 - Wert 
	    \item rekursiver Aufruf, bis alle Array Positionen besetzt sind
    \end{itemize}
    \end{frame}

	\begin{frame}{Unrank \newline Part 3: Beispiel}
	Übergebener Index: $i = 23$, $q=3, length=3$ \newline
	
	Bereiche: $0-8, 9-17, 18-26$ \newline
	Schreibe 2 (gerade, keine Anpassungen); unrank(23 - 18 = 5, nicht inventieren) \newline
	
	Bereiche: $0-2, 3-5, 6-8$ \newline
	Schreibe 1 (ungerade, inventiere folge Rekursionen); unrank(5 - 3 = 2, inventiere) \newline
	
	Bereiche: $0, 1, 2$ \newline
	Eigentliche Rückgabe: $2$  \newline
	inventieren: $3 - 1 - 2 = 0$
	\end{frame}    
    
    
\section{Successor und Predecessor}
    \begin{frame}{Successor und Predecessor \newline Part 1: Allgemeines}
    Allgemeines:
    \begin{itemize}
	    \item Dienen zur Bestimmung des nächsten (vorherigen) Codes
	    \item Benötigt wissen über das Alphabet $q$
	    \item (Benötigt wissen über die Länge des Codes)
    \end{itemize}
    \end{frame}
    
    \begin{frame}{Successor und Predecessor \newline Part 2: Algorithmenbeschreibung - Langsame Lösung}
    Algorithmus:
    \begin{itemize}
	    \item Hole den Index des Objektes mit der \textit{Rank-Funktion}
	    \item Addiere/Subtrahiere 1 von dem erhaltenen Index
	    \item Generier ein neues Objekt mit der \textit{Unrank-Funktion}
    \end{itemize} 
    Performanzproblem:
	\begin{itemize}
	    \item Generierung eines neuen Objektes \newline => vorzüge von \textit{minimal-change} gehen \say{verloren}
    \end{itemize}
    \end{frame}
        
    \begin{frame}{Successor und Predecessor \newline Part 3: Algorithmenbeschreibung - Effizient}
    Algorithmus:
    \begin{itemize}
	    \item Rufe Ink-/Dek-rementiere auf letztes Element auf
	    \item Wiederhole dies bis zum ersten Element \newline Bei Ungeraden Eigenwert: inventiere den Aufruf \newline (ink wird zu dek und umgekehrt)
	    \item Kann das erste Element nicht in/de krementieren \newline Versuche das zuvor kommende Element
    \end{itemize}
    \end{frame}
    
	\begin{frame}{Successor und Predecessor \newline Part 4: Beispiel 1 - Successor}
    Übergebenes Objekt: $succ(012)$, $q=3$ \newline
    
    ink(012) 2 -> ink(01)\newline
    ink(01)  1 -> dek(0) \newline
    dek(0)   nicht möglich -> return false \newline
    ink 1 möglch -> schreibe 2, return true \newline
    
    Ergebnis: \newline
    022
    \end{frame}

	\begin{frame}{Successor und Predecessor \newline Part 4: Beispiel 2 - Predecessor}
    Übergebenes Objekt: $pred(012)$, $q=3$ \newline
    
    dek(012) 2 -> dek(01)\newline
    dek(01)  1 -> ink(0) \newline
    ink(0) möglich -> schreibe 1, return true \newline

    Ergebnis: \newline
    112
    \end{frame}


\section*{Haben Sie Fragen?}
\section*{Vielen Dank für die Aufmerksamkeit!}
  
\end{document}
